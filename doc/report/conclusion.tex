In this paper, we presented Recoverable Virtual Memory, our solution for easy
replication and recovery of application state. We have attempted to provide our
implementation with features we believe are highly desirable for the
application programmer, such as a simple and understandable API; restoration of
virtual address locations, which allows for recovery of complex data structures
without the need for serialization; and reasonably low overhead.  Of these
three goals, we have accomplished the first two, but there is still
considerable work to be done in regards to performance.

The most obvious optimization we could make is to coalesce RDMA operations for
contiguous pages. In our current infiniband backend, allocation of the backing
memory for multiple contiguous pages must be done a page at a time.  However,
it would be much more efficient to perform a single contiguous allocation on
the server side. That way, the number of RDMA write calls and TXN\_MULTI\_CP
messages does not need to increase as the number of pages increases. There are
probably other areas for performance improvement that we could discover through
more intensive profiling of the benchmark programs.

Another avenue we could explore are alternative backends for the RMEM layer.
There are various networking and non-volatile memory technologies that we could
investigate, such as SSDs and RDMA over Converged Ethernet. We could also
implement different consistency semantics to explore the tradeoffs of
performance and consistency.

Finally, we would like to integrate RVM with existing runtimes and recovery
frameworks to provide a more complete data replication and recovery solution.

The complete code for our RVM implementation is available on GitHub at
https://github.com/zhemao/rmem-server.
