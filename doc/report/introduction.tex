As the number of nodes in distributed systems increases, failures become the
rule, not the exception. Because of this, it is important to be able to recover
from crashes quickly.  In addition, the application should not be made
significantly more complicated by the addition of recovery code.

While it would be overly ambitious to try and solve this entire goal, we propose
a tool that could simplify such a solution by providing the ability to replicate
the application's working state to another failure domain. In this report, we
present an implementation of Recoverable Virtual Memory (RVM), an API that
allows the programmer to easily add state checkpointing and recovery to their
application. RVM operates by detecting modifications to recoverable memory
regions and replicating the memory to a remote node.

The proliferation of high-performance RDMA and future disaggregated memory
systems offer an opportunity to perform this replication efficiently. For
example, the FireBox warehouse-scale computer (an ASPIRE lab project) will have
a central pool of universally accessible DRAM and non-volatile memory. Currently
there is no FireBox hardware to experiment with, but we do have an infiniband
based cluster (Firebox-0). So far, we have implemented two backends, one using
the Infiniband RDMA API and one using the RAMCloud key-value store.

\subsection{Current Solutions}
The need to preserve critical data in the event of a hardware failure is not
new. There are a number of popular methods of addressing this problem. One is to
checkpoint the entire operating system process using a tool like the Berkeley
Lab Checkpoint Restart library\cite{BLCR} or Condor\cite{Condor}). Process
checkpointing is appealing because it requires little to no changes in the
application. For this convenience, the technique sacrifices efficiency. All data
must be checkpointed, not just the critical state, and operating system state
must be quiesced. A common practice to avoid avoid copying the entire program
state is to manually serialize critical data structures and write them to a
file. While, in theory, this technique copies the minimum amount of data, in
practice it can be difficult to identify which state has actually changed. The
user is forced to pessimistically replicate most critical state on each
checkpoint. Manual serialization also leads to significant increases in code
complexity. Each data structure must have two definitions, one for on-disk and
the other for in-memory. Maintaining this serialization code can be
time-consuming and error-prone. Finally, databases are commonly used to
replicate critical state.
Databases provide clean transactional semantics that can be appealing for
high-availability applications. Databases, however, often have a complex
interface that requires manual serialization. They also provide more features
than are required for state replication, which leads to poor performance.

\subsection{RVM Interface}
The RVM interface is designed to be as unobtrusive as possible. Users should be
able to preserve just their critical state without worrying about re-writing
pointers or packing data into a file. To do this our framework requires only
that the user identify which memory is considered critical, and identify points
of consistency in their code. By marking a point of consistency the user
certifies that, if the program we're to restart with the critical memory in the
current state, the program would be able to continue execution. Critical state
is identified by allocating it from a special recoverable malloc function and
consistency points are identified through a transactional interface. Users may
also save a special {\bf user\_data} pointer that survives failures. This pointer
typically stores a state structure that can address the recoverable state in an
application-dependent fashion. Table~\ref{tab:rvm_api} lists the entire required API.

\begin{table}[t!]
\centering
\caption{RVM API}
\label{tab:rvm_api}
\resizebox{\columnwidth}{!}{%
   \begin{tabular}{ | l | l | }
     rvm\_cfg\_[create/destroy]() & \pbox{20cm}{Initialize the system \\and recover memory if
     needed}
     \\
     \hline rvm\_[alloc/free]() & \pbox{20cm}{Allocate a region \\of recoverable memory} \\
     \hline rvm\_txn\_[start/commit]() & \pbox{20cm}{Mark a point of \\consistency in the program}
     \\
     \hline rvm\_[set/get]\_usr\_data() & \pbox{20cm}{Register a pointer \\to your state.} \\
     \end{tabular}
   }
\end{table}

In practice, of course, it is not this simple. Code must be written in such a
way that recovery is possible. In addition, while critical local state is
preserved, external state (like open files or sockets) is not. RVM is intended
as a low-level library that can be exploited by more full-featured recovery
libraries. An analogous relationship can be seen in the GASNet \cite{GASNET}
library which can be used directly, but is really intended as a low-level
interface for global address space languages.
