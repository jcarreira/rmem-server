% DGEMV
Our first attempt at making a realistic application recoverable was an iterative
dense matrix-vector multiply (DGEMV) code. This application repeatedly
multiplies a vector by a constant matrix. Since the matrix is constant, the
only state that needs to be preserved is the vector and the last successful
iteration. This makes DGEMV somewhat of a best-case scenario for traditional
recovery schemes. The state is a single large allocation that changes entirely
on each iteration. In the following experiments, our RVM implementation is
compared with a manual serialization scheme that writes to SSD's on our
experimental system. The matrix was chosen to be of dimension 1Mx100 with 100
iterations in order to maximize the critical state and stress the recovery
schemes.

%XXX Graphs go here

In figure \ref{DGEMV Commit rate} we vary the commit rate (in terms of
iterations) from 1 (every time) to 100 (only one commit) without failures. This
essentially measures the overhead caused by recovery. In all cases, RVM
introduces less overhead. This is likely due to the efficient nature of RVM
checkpoints. More interesting, however, is when we introduce failures. In figure
\ref{DGEMV Failure rate}, we always commit on each iteration, but inject
failures after different numbers of iterations (from each iteration up to a
single failure). In this case, the file-based backend is faster than RVM for
high failure rates, but slower with low failure rates. We believe this is due to
a higher startup cost for the file-based approach. It also demonstrates that
reading a relatively large, sequential file from an SSD is very performant and
competes will with RDMA. We believe that further optimizations in the RMEM layer
could make up some of this difference, see \ref{future work} for some possible
performance enhancements.
