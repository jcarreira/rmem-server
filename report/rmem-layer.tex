Underlying the RVM API is the remote memory (RMEM) layer, which provides the
basic operations that RVM uses to communicate with the backing data store.
The essential operations in the RMEM layer are \texttt{malloc()},
\texttt{free()}, \texttt{put()}, \texttt{get()}, and \texttt{atomic\_commit()}.

The \texttt{malloc()} function allocates memory in the backing store.
The function arguments include the number of bytes to allocate, as well as
a unique tag that is associated with that memory region. If the tag has not
already been taken, \texttt{malloc()} allocates a new memory region and returns
the starting address. If a memory region with that tag already exists, the
starting address of the previously allocated region is returned.

The \texttt{free()} function takes a tag as its argument and frees the memory
region associated with that tag.

There are also \texttt{multi\_malloc()} and \texttt{multi\_free()} functions
which allocate and free multiple memory blocks. Depending on the backend,
these functions might coalesce \texttt{malloc()} and \texttt{free()} requests
in order to decrease the number of round trips to the backing store.

The \texttt{put()} and \texttt{get()} functions copy data to and from the
backing store, respectively. These operations are not atomic, so the RVM
layer always performs puts and gets onto a shadow page and then copies the
data to the real page using \texttt{atomic\_commit()}.

As mentioned before, \texttt{atomic\_commit()} takes an array of source tags
and an array of destination tags. It instructs the server to copy the data
in the source memory regions to the destination regions. This copying is
done atomically, so there is no danger of only a portion of the pages
being copied due to the client crashing.
