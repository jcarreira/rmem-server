The RVM layer is responsible for implementing the custom allocator and
identifying changes to memory. It is also responsible for tracking commit
points.

\subsubsection{The Block Table}
The RVM layer thinks in terms of \emph{blocks}, fixed-sized regions of
memory that are persisted atomically. Most functionality is based around the
\emph{block table}, a persistent data structure that keeps track of each
allocated block in the system. This table is replicated using the same
 mechanims as any other recoverable memory. Like a filesystem master boot record,
 the first page of the block table is always
stored with a constant identifier in the RMEM layer. After that, the block table
is self-describing and can be recovered using the mechanisms described below.

Each entry in the block table contains a local address where the active block
lives on the client, and a remote identifier that can be used to identify the
block in the RMEM layer. 

\subsubsection{Initialization and Recovery Procedure}
When \verb|rvm_cfg_create()| is called the first time, it initialized the
block table to an empty state and persists it in the RMEM layer. When
recovering, \verb|rvm_cfg_create()| fetches the first block of the block table from the
RMEM layer. The block table is walked from start to finish, fetching each block as it goes. Even if the block table takes up multiple blocks, each one is fetched in order, ensuring that all data
can be found eventually. When RVM fetches a remote block, it must ensure that it
is loaded to the same address it was at before failure, otherwise pointers in
the data would no longer be valid. The original address is read from the block
table and then allocated using the \emph{mmap()} system call. To ensure that
these addresses are always available, RVM requires that any OS address space
layout randomization be disabled, and that rvm_cfg_create() be called before any
other local allocations.

\subsubsection{Allocation}
To ensure that memory is recoverable, the user must allocate it using a special
\verb|rvm_alloc()| function. The \verb|rvm_alloc()| functions allocates memory both locally and on the remote
node. Any modifications to the local pages allocated by \verb|rvm_alloc()| are
automatically detected and copied to the remote node at commit time. Detection
is acheived through the use of \emph{mprotect()} a Linux system call that
can be used to make the application take an interrupt whenever a page is
written. Our custom interrupt handler then marks the page as changed, removes
the memory protection and returns. This means that RVM needs to be involved only
in the first modification to a page.

\subsubsection{Marking a Point of Consistency}
The user is required to identify points in their code where the state of
recoverable memory is considered \"consistent\". This means that recovery is
possible from that particular state. \verb|rvm_txn_commit()| can be called at
these points to ensure that memory is atomically persisted. Upon entering
\verb|rvm_txn_commit()|, RVM goes through the list of changed pages and coppies
them to a shadow page in the RMEM layer. This ensures that consistent version
of memory is always available, even if the client crashes during checkpointing.
When all the pages have been coppied, an \verb|atomic_multi_copy()| function
(provided by the RMEM layer) is called to persist the changes.

