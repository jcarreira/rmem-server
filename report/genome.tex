% Genome Assembly
For a more complex benchmark, we ported the de novo genome assembly code that
was the basis of homework 3. This code involves the use of a large hash-table
(rough 100MB with our test input) as well as several linked-lists and other
complex data structures. In some ways, this is a best-case scenario for RVM. On
each checkpoint, only a fraction of the data is changed, and much of the memory
is constant (the input file). While we initially intended to implement a manual
serialization as we had done for DGEMV, it quickly became apparent that such an
undertaking would be almost as involved as writing the application in the first
place. We instead decided to compare it against a popular process-level
checkpointer called BLCR\cite{BLCR}. While DGEMV needed only to store the vector
and an iteration number, the genome assembly code was considerably more
complicated. Execution consists of two main phases. In the build phase, kmers
are read from a file and inserted into a hash table. The probe phase probes into
this hash table in order to construct the final contigs. To capture this
pattern, a global state structure was allocated from recoverable memory. This
global state stores shared structures such as the hash table and start-kmer
list, which phase is currently being executed (build or probe), and then
provides an opaque phase state pointer to be filled in by phase-specific code.
 The build phase keeps track of where in the input file it was, while probes
state is more complex. It needs to preserve which start-kmer it was
processing, where in the current contig it was, and where in the output file it
was writing. Each phase processed all 4 million kmers, for a total of 8 million
processing steps.

In figure \ref{GEN COMMIT RATE} we vary the commit rate (in terms of number of
processing steps) from every 10 thousand (out of 8 million) up to 4 million
(one checkpoint per phase). At higher commit rates, RVM clearly outperforms
BLCR. In the exreme, RVM was able to complete in 800 seconds with a commit rate
of 10K, while we were forced to cancel the BLCR run after 30 minutes. Even at
more reasonable commit rates such as 1 million (8 commits), RVM was nearly twice
as fast as BLCR. It's only at the extreme low end of commit rates (4 million
and no commit) that BLCR was faster. This is because RVM pays a significant
up-front cost to allocate it's large recoverable state. Further analysis showed
that of the 10 seconds spent at a 4 million commit rate, a full 5s were spent
simply allocating memory, while build took only 2s and probe only 1s. This
clearly indicates a performance bottlenect in our system, although we do not
believe it is fundemental to the approach. See \ref{future work} for a
discussion of possible performance improvements. 
