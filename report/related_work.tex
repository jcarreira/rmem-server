\paragraph {\bf Virtual machine / Container checkpoint} Systems such
as~\ref{Tardigrade} or VMWare provide some level of fault-tolerance by
checkpointing containers and virtual machines, respectively. While these
systems can checkpoint a program's data without knowing the program's internals
they have limitations. First, using virtual machines incurs a performance
overhead on applications. Secondly, checkpointing an entire virtual machine or
container can be expensive. We believe RVM provides an API that can be used to
checkpoint only the data that matters to the user, and thus it can provide
better performance with minimal developer effort.

\paragraph {\bf Key-value stores and DBMSs} Key-value stores such as RAMCloud
and DBMSs such as Postgres can be used to store a program's data to remote nodes
and provide similar properties as RVM. While we believe that many of the
techniques and lessons used in these systems can be applied in RVM, we think
these systems are not a good fit for the applications we envision using with
RVM. First, most modern key-value stores do not provide atomic multi-key
writes. Secondly, the API provided by key-value stores despite being simple is
not suitable for virtual memory replication. Likewise, DBMSs sacrifice data
access latency in favour of many features that are not required in this context
while. Additionally, the schema model required by traditional DBMSs does not fit
well with arbitrarily-sized regions of memory.

\paragraph {\bf Virtual memory replication} Systems like Mojim~\cite{Mojim} or
LRVM~{LRVM} can be used to replicate virtual memory.
